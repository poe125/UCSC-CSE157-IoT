%Setting the latex style wanted 
\documentclass{article}

% Packages to be used
\usepackage{hyperref}       %Used for hyperlinks 
\usepackage{xcolor}         %Used for code formatting
\usepackage{listings}       %Used for code formatting
\usepackage{lstautogobble}  %Used to help listing package
\usepackage{float, graphicx}       %Used for inserting images
    \graphicspath{{./images/}}     %Path to images

%Setting code formatting
\lstset{basicstyle=\ttfamily,
  showstringspaces=false,
  commentstyle=\color{red},
  keywordstyle=\color{blue},
  breaklines = true, 
  autogobble=true
}

% Adding pre-amble stuff but not activated 
\title{Lab 1 Write Up}
\author{Keith Irby}


% Part 1: Setup Instructions
\begin{document}
    \maketitle %activate title
    \section{Setup Instructions}
    This part describes how to setup the Raspberry Pi in detail using UCSC Eduroam wifi and updating the Raspberry Pi
    
    % Part 1: Setup Instructions itemized list
    \begin{enumerate}
        \item Download the \href{https://ubuntu.com/tutorials/how-to-install-ubuntu-desktop-on-raspberry-pi-4#2-prepare-the-sd-card}{Raspberry Pi Imager}
        \item Install the latest 32 bit version of Raspberry Pi onto a SD card (Used release: 2023-05-03 )
        \item Plug the SD card you just flashed into the Raspberry Pi along with power, video out, mouse, and keyboard
        \item Let the Raspberry Pi boot and follow the setup instruction. Skip wifi setup and restart the Pi. 
        %\begin{figure}
            %\centering
        %    \includegraphics[width=\linewidth]{0_boot_up_screen.png}
        %    \caption{Raspberry Pi OS boot up Screen}
        %\end{figure}
        \item Once restarted you should be greated by the boot screen 
        \item Copy the wifi.sh file over to the Raspberry Pi 
        \item Open a terminal and go to the same directory as the wifi.sh file
        \item run the following command on the Raspberry Pi wifi.sh file to give it edit access: 
            %Adding a terminal command           
            \begin{lstlisting}[language=bash]
            rpi@raspberrypi:~ $ chmod +x wifi.sh
            \end{lstlisting}
        \item Run the provided wifi.sh script using your Cruz ID gold login (username not email and gold password) 
            %Adding terminal command           
            \begin{lstlisting}[language=bash]
            rpi@raspberrypi:~ $ ./wifi.sh
            \end{lstlisting}
        \item Install VIM because it is better than Nano using the following command: 
            %Adding a terminal command
            \begin{lstlisting}[language=bash]
            rpi@raspberrypi:~ $ sudo apt install vim
            \end{lstlisting}
        \item Update the Raspberry Pi by running the following command:
            %Adding a terminal command
            \begin{lstlisting}[language=bash]
            rpi@raspberrypi:~ $ sudo apt-get update
            \end{lstlisting}
        \item Reboot the Raspberry Pi by running the following :
            %Adding a terminal command
            \begin{lstlisting}[language=bash]
            rpi@raspberrypi:~ $ sudo reboot
            \end{lstlisting}
    \end{enumerate}
    \subsection{Notes and Comments}
    If the current method isnt working for you to setup a connection to eduroam on UCSC 
    campus use the \href{https://its.ucsc.edu/wireless/eduroam-manual-config.html}{following guide} to do it. Also, 
    if you want to change your password in the future you can use the following command to open a menu to change it:
    \begin{lstlisting}[language=bash]
        rpi@raspberrypi:~ $ sudo  raspi-config
        \end{lstlisting}


    \section{Setting up and connecting to SSH (Windows 11)}
    \begin{enumerate}
        \item Boot up the Raspberry Pi 
        \item open the terminal and run the following command: 
        %Adding a terminal command
        \begin{lstlisting}[language=bash]
            rpi@raspberrypi:~ $ sudo raspi-config
            \end{lstlisting}
        \item navigate to interfaces and turn on SSH 
        \item Run the following command to get the Raspberry Pi's IP Address for SSH:
        %Adding a terminal command
            \begin{lstlisting}[language=bash]
            rpi@raspberrypi:~ $ hostname -I
            \end{lstlisting}
        \item Install \href{https://www.chiark.greenend.org.uk/~sgtatham/putty/latest.html}{PuTTY} for Windows 11
        \item Open the PuTTY client and set the \textbf{Host Name} to the IP address you just found on the Raspberry Pi and the \textbf{Port} as 22
        \item Click \textbf{Open} on the bottom of the PuTTY application and login using your Raspberry Pi Credentials
    \end{enumerate}
    \subsection{Notes and Comments}
    I would note using SSH might not be very useful for the CSE 157 course and do not see myself using this. I dont find it 
    useful because when the Pi is in Ad-Hoc mode you cannot SSH into it without being nearby for a connection, which from what I understand about the course 
    the Pi will be in Ad-Hoc mode a decent amount of the time. Also, I find Vim easier to use than Nano or VS code for quick development 
    on smaller projects like this. Vim over nano because I was taught Vim first. Vim over VS code because when you're working with embedded 
    devices its always quicker to develop without the bloat of an entire IDE. 

    %
    % Setting up ad-hoc mode
    %
    \section{Ad-hoc Mode and Wifi Mode}
    \subsection{Ad-Hoc Mode}
    \subsubsection{Preparing for Ad-hoc mode}
    \begin{enumerate}
        \item Open the terminal and install the following package like so: 
        %Adding a terminal command
        \begin{lstlisting}[language=bash]
            rpi@raspberrypi:~ $ sudo apt-get install isc-dhcp-server 
        \end{lstlisting}
        \item Next go to the following directory using the terminal like so: 
        %Adding a terminal command
        \begin{lstlisting}[language=bash]
            rpi@raspberrypi:~ $ cd /etc/network 
        \end{lstlisting}
        \item After getting to  directory \verb|cd /etc/network| use the following command to make a copy of the current wifi interface: 
        %Adding a terminal command
        \begin{lstlisting}[language=bash]
            rpi@raspberrypi:~ $ sudo cp interfaces wifi-interface 
        \end{lstlisting}
        \item Now make a new Ad-Hoc interface like so: 
        %Adding a terminal command
        \begin{lstlisting}[language=bash]
            rpi@raspberrypi:~ $ sudo vim adhoc-interface 
        \end{lstlisting}
        \item Then copy the following code contents into the newly created Ad-Hoc interface: 
        %Adding a terminal command
        \begin{lstlisting}[language=bash]
            auto lo
            iface lo inet loopback
            iface eth0 inet dhcp
          
            auto wlan0
            iface wlan0 inet static
            address 192.168.1.1
            netmask 255.255.255.0
            wireless-channel 4
            wireless-essid RPitest
            wireless-mode ad-hoc
        \end{lstlisting}
        \item Then open the following file using the following command:
        %Adding a terminal command
        \begin{lstlisting}[language=bash]
            rpi@raspberrypi:~ $ sudo vim /etc/dhcp/dhcpd.conf 
        \end{lstlisting}
        \item Once the \verb |dhcpd.conf| file is open add the following to the very bottom of the file with no comments: 
        %Adding a terminal command
        \begin{lstlisting}[language=bash]
            ddns-update-style interim;
            default-lease-time 600;
            max-lease-time 7200;
            authoritative;
            log-facility local7;
            subnet 192.168.1.0 netmask 255.255.255.0 {
             range 192.168.1.5 192.168.1.150;
            }
        \end{lstlisting}
        \item Reboot and follow the subsection below to start Ad-Hoc mode, if you want it to start in WiFi mode do the following to swap interfaces back to wifi: 
        %Adding a terminal command
        \begin{lstlisting}[language=bash]
            rpi@raspberrypi:~ $ sudo cp /etc/network/wifi-interface /etc/network/wifi-interface/interfaces 
        \end{lstlisting}
    \end{enumerate}
    \subsubsection{Starting Ad-Hoc mode}
    \begin{enumerate}
        \item Open a terminal
        \item   to turn off the networking client:
            %Adding a terminal command
            \begin{lstlisting}[language=bash]
            rpi@raspberrypi:~ $ wpa_cli terminate
            \end{lstlisting}
        \item Type the following command to turn off the networking client:
            %Adding a terminal command
            \begin{lstlisting}[language=bash]
            rpi@raspberrypi:~ $ sudo ifconfig wlan0 down
            \end{lstlisting}
        \item Type the following command to turn the Raspberry Pi into Ad-Hoc mode replacing \verb!<ssid>! with what you want the network to be named:
            %Adding a terminal command
            \begin{lstlisting}[language=bash]
            rpi@raspberrypi:~ $ sudo iwfconfig wlan0 mode ad-hoc channel 04 essid <ssid>
            \end{lstlisting}
        \item Type the following command to start the Ad-Hoc mode:
            %Adding file contents
            \begin{lstlisting}[language=bash]
                rpi@raspberrypi:~ $ sudo ifconfig wlan0 up
                \end{lstlisting}
    \end{enumerate}
    \subsection{Wifi Mode}
    \begin{enumerate}
        \item First do the following command: 
        %Adding a terminal command
        \begin{lstlisting}[language=bash]
            rpi@raspberrypi:~ $ sudo cp /etc/network/wifi-interface /etc/network/interfaces 
        \end{lstlisting}
        \item  Then use the following command:
         %Adding a terminal command
         \begin{lstlisting}[language=bash]
            rpi@raspberrypi:~ $ systemctl restart networking
        \end{lstlisting}
        \item Then type the following command to restart dhcpcd services:
        %Adding a terminal command
        \begin{lstlisting}[language=bash]
            rpi@raspberrypi:~ $ systemctl restart dhcpcd
        \end{lstlisting}
        \item Finally, reboot the system with the following command:
        %Adding a terminal command
        \begin{lstlisting}[language=bash]
            rpi@raspberrypi:~ $ sudo reboot 
            \end{lstlisting}
    \end{enumerate}


    \section{Generating a weather report}
    \begin{enumerate}
        \item Install python requests package using the following command:
        %Adding a terminal command
        \begin{lstlisting}[language=bash]
            rpi@raspberrypi:~ $ pip3 install requests
        \end{lstlisting}
        \item run the following command on the python script provided: 
        %Adding a terminal 
        \begin{lstlisting}[language=bash]
            rpi@raspberrypi:~ $ python3 generate_report.py
        \end{lstlisting}
    \end{enumerate}
    \subsection{Notes and Comments}
    If you want to see a description on how the \verb|generate_report.py| works, open the file. I added comments throughout it 
    and used functions so it is easier for newer developers to understand what is happening. Also, the program uses the open-meteo
    own api which is just a HTTPS get request containing data from their API editor found \href{https://open-meteo.com/en/docs}{here}. You configure 
    what you want in your API call by clicking check boxes and typing the location using latitude and longitude, then copy API URL found on the page, 
    use it like in \verb|generate_report.py|.

    \section{Adding weather report generation to bootup}
    \begin{enumerate}
        \item Run the following command to open a file that runs on boot:
        \begin{lstlisting}[language=bash]
            rpi@raspberrypi:~ $ sudo vim /home/rpi/.bashrc
        \end{lstlisting}
        \item At the very bottom of this file add the following line, it should be noted the path is whereever you placed \verb|generate_report.py|:
        \begin{lstlisting}[language=bash]
            rpi@raspberrypi:~ $ python3 ~/Documents/lab1/generate_report.py
        \end{lstlisting}
    \end{enumerate}


    \section{Conclusion}
    From this lab I learned setting up a Pi into Ad-Hoc mode is confusing and could not 
    find a way to get it working on my own. I also could not find a resource for the code 
    the TA provided for turning the Pi into Ad-Hoc mode. I did learn how to insert a bootup 
    script with a python file, which I didnt know was so easy. Other than this I have developed with 
    python, used Linux distros, and done API calls all in the past.
    \subsection{sources}
    \begin{enumerate}
        \item \href{https://roboticsbackend.com/install-use-vim-raspberry-pi/}{Source 1} 
        \item \href{https://tex.stackexchange.com/questions/162323/how-to-left-align-lstlisting-code-with-text}{Source 2}
        \item \href{https://www.onlogic.com/company/io-hub/how-to-ssh-into-raspberry-pi/}{Source 3}
        \item \href{https://pyshine.com/How-to-configure-Raspberry-Pi-in-Ad-hoc-wifi-mode/}{Source 4}
        \item \href{https://www.geeksforgeeks.org/get-current-date-and-time-using-python/}{Source 5}
        \item \href{https://realpython.com/python-requests/}{Source 6}
        \item \href{https://www.geeksforgeeks.org/python-datetime-module/}{Source 7}
        \item \href{https://www.geeksforgeeks.org/writing-to-file-in-python/}{Source 8}
        \item \href{https://stackoverflow.com/questions/35415647/how-to-write-a-file-to-the-desktop-in-python}{Source 9}
    \end{enumerate}


\end{document} 


%Resources
%part 1 setup resources 
% https://roboticsbackend.com/install-use-vim-raspberry-pi/
% https://tex.stackexchange.com/questions/162323/how-to-left-align-lstlisting-code-with-text
%Part 2 ssh resources
%https://www.onlogic.com/company/io-hub/how-to-ssh-into-raspberry-pi/
%part 3 ad-hoc mode resources 
%
%part 4 resources



Volume of Base (mL) pH  Volume of Base (mL) pH Volume of Base (mL)  pH
                
                
                
                
                
                
                
                
                
                
                